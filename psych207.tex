\documentclass[]{article}
\usepackage[margin = 1.5in]{geometry}
\setlength{\parindent}{0in}
\usepackage{amsfonts}
\usepackage{amssymb}
\usepackage{hyperref}
\usepackage[T1]{fontenc}
\usepackage{ae,aecompl}

\setlength{\marginparwidth}{1.5in}
\newcommand{\lecture}[1]{\marginpar{{\footnotesize $\leftarrow$ \underline{#1}}}}

\begin{document}
	
	\title{\bf{PSYCH 207: Cognitive Processes}}
	\date{Winter 2013, University of Waterloo \\ \center Notes written from Jonathan Fugelsang's lectures.}
	\author{Chris Thomson}
	\maketitle
	\newpage

	\section{Introduction \& Course Structure} \lecture{January 8, 2013}
		\subsection{Course Structure}
			The grading scheme is four in-class non-cumulative multiple-choice exams, equally weighted. There is also a 4\% bonus for research participation through SONA. You should get the textbook.
			\\ \\
			See the course syllabus for more information -- it's available on \href{https://learn.uwaterloo.ca/}{Waterloo LEARN}.
		
		\subsection{Introduction to Cognitive Processes}
			\begin{quote}
				``Cognitive psychology refers to all processes by which the sensory input is transformed, reduced, elaborated, stored, recovered, and used." \textendash{}  Neisser, 1967
			\end{quote}
			\textbf{Cognitive psychology} involves perception, attention, memory, knowledge, reasoning, and decision making.
			\\ \\
			\textbf{Cognitive processes} are everything that goes on in our mind that affects our environment. Many of these processes are completely unconscious.
			\\ \\
			Conscious experience is an \underline{active reconstructive process}. The external world and our internal representation of that world is \emph{not} an exact match. Our brain ends up filling in many gaps, making many assumptions.
			\\ \\
			Our brain cannot decontextualize the world.
			
	\section{Historical Overview \& Approaches} \lecture{January 10, 2013}
		\underline{Attention} (notice \emph{something}) $\to$ \underline{Perception} (perceive that \emph{something}) $\to$ \underline{Pattern Recognition} (recognize what that \emph{something} is) $\to$ \underline{Memory} (recall previously-known attributes about the \emph{something}).
		\\ \\
		Our cognitive apparatus is ultimately an efficient simplification process.
		
		\subsection{Antecedent Philosophies and Traditions}
			Many researchers take very strong views on empiricism vs. nativism, however reality is most likely somewhere between the two. The debate for structuralism vs. functionalism is similar.
		
			\subsubsection{Empiricism}
				\begin{itemize}
					\item Locke, Hume, and Stuart Hill.
					\item Emphasis is on experience and \underline{learning}.
					\item Key is the \underline{association} between experiences.
					\item This is observational learning -- the nurture side of the nature vs. nurture argument.
				\end{itemize}
				
			\subsubsection{Nativism}
				\begin{itemize}
					\item Plato, Descartes, and Kant.
					\item Emphasis is on that which is \underline{innate}.
					\item Innate causal mechanisms.
					\item This is the nature side of the nature vs. nurture argument.
				\end{itemize}
				
			\subsubsection{Structuralism}
				\begin{itemize}
					\item Wundt and Baldwin.
					\item The focus is on the elemental components of mind.
					\item Very reductive -- it's about stripping out context to understand the very basic elements.
					
					\item \underline{Introspection} (method)
						\begin{itemize}
							\item Report on the basic elements of consciousness.
							\item Not internal perception, but \underline{experimental self observation}.
							\item Must be done in a lab under controlled conditions.
							\item Basic elements of the conscious experience include processes like identifying colors.
						\end{itemize}
				\end{itemize}
			
			\subsubsection{Functionalism}
				\begin{itemize}
					\item William James.
					\item Regarded the mission of psychology to be the explanation of our experience.
					\item Key question: why does the mind work as it does?
					\item The function of our mind is more important than its content.
					\item \underline{Introspection in natural settings} (method)
						\begin{itemize}
							\item Must study the whole organism in real-life situations.
							\item Must get out of the lab to conduct functionalist research.
						\end{itemize}
				\end{itemize}
			
			\subsubsection{Behaviorism}
				\begin{itemize}
					\item Watson and Skinner.
					\item Started in the 30s and was the dominate focus of academic psychology until the 60s.
					\item Originally evolved as a reaction to the lack of progress provided by introspection.
					\item A behaviorist sees psychology as an objective, experimental branch of science. Psychology's goal is the prediction and control of behaviour. Therefore, they make behavior (not consciousness) the focus of their research.
					\item Focus is on the relation between input and output, but the steps in between (which make up cognitive psychology) do not matter to behaviorists.
				\end{itemize}
				
			\subsubsection{Gestalt Psychology}
				\begin{itemize}
					\item Wertheimer, Koffka, and Kohler.
					\item Focus is on the holistic aspects of conscious experiences.
					\item Key question: what are the rules by which we parse the world into wholes?
					\item \underline{Introspection} (method)
						\begin{itemize}
							\item Experience is simply described, never analyzed.
						\end{itemize}
					\item A unified whole is often different than the sum of its parts. How do we impose structure on what's already out there? For example: 8 line segments in groups of 2 are interpreted differently than the 8 lines being all scrambled together in a seemingly random way.
					\item How does the mind simplify the world to focus our attention on things/objects that matter?
					\item We need to study phenomena in their entirety, since a unified whole is different than the sum of its parts.
				\end{itemize}
			
			\subsubsection{Individual Differences}
				\begin{itemize}
					\item Sir Francis Galton.
					\item Intelligence, morals, and personality are innate.
					\item Mental imagery was studied in both a lab and in natural settings. The vividness of mental imagery differs from person to person.
					\item Galton invented the process of using questionnaires to assess abilities. This process has been used by cognitive psychologists ever since.
				\end{itemize}
			
		\subsection{The Cognitive Revolution}
			\begin{itemize}
				\item The speed of information publishing, sharing, and retrieval has become very fast.
				\item We're now running into the cognitive speed limit as our limiting factor, whereas before communication channels (snail mail, travel) slowed down research.
				\item Recent advances in neuroimaging are also a mini-revolution in cognitive psychology.
			\end{itemize}
			
			\subsubsection{Human factors engineering presented new problems}
				\begin{itemize}
					\item A machine should be designed for human use -- for use in the most efficient way possible. Knowledge of human cognition is required in order to increase efficiency.
					\item We have to think about the $7 \pm 2$ information limitation of the human mind, and how to get around the limit.
					\item NASA hires cognitive psychologists to study how the human mind operates in extreme conditions. Cognitive psychologists develop the user interfaces that astronauts use.
				\end{itemize}
			
			\subsubsection{Behaviorism failed to adequately explain language}
				\begin{itemize}
					\item Skinner in 1957 (behaviorism): children learn language by imitation and reinforcement.
					\item Chomsky in 1959 questioned Skinner's explanation of language.
						\begin{itemize}
							\item Children often say sentences they have never heard before, such as ``I hate you mommy.'' (Not imitation.)
							\item Children often use incorrect grammar, such as ``The boy hitted the ball'', despite a lack of reinforcement.
						\end{itemize}
				\end{itemize}
				
			\subsubsection{Localization of functions in the brain forced discussion of mind}
				\begin{itemize}
					\item Donald Hebb stated that some functions, like perception, are based on cell assemblies (collections of neurons).
					\item Hubel and Weisel demonstrated the importance of early experiences on the development of the nervous system. Early experiences actually change how some cell assemblies physically develop.
					\item Many things seem to happen without observational learning coming into play.
				\end{itemize}
				
			\subsubsection{Development of computers and artificial intelligence gave a dominant metaphor}
				\begin{itemize}
					\item A computer takes input into short-term memory (RAM), may acceess long-term memory (a hard drive), and returns some output.
					\item The mind may work in a similar way.
					\item Perhaps we introspected and that's why we developed computers the way we did?
				\end{itemize}
				
		\subsection{Paradigms of Cognitive Psychology}
			\begin{itemize}
				\item Emphasis is on serial processing.
				\item Information is stored symbolically.
				\item The mind is an information processing system with systems of interrelated capacities.
				\item All of these attributes are similar to that of typical computer systems.
			\end{itemize}
			
			\subsubsection{Localist models}
				\begin{itemize}
					\item A symbolic concept, such as a letter, word, or meaning, is represented in your mind with a node.
					\item You may have a node for `cat', `dot', or `house' (lexical knowledge). You may also have a node for `provides shelter', `barks', or `has four legs', all of which are boolean attributes (semantic knowledge).
				\end{itemize}
				
			\subsubsection{Connectionism -- Neural network models}
				\begin{itemize}
					\item Parallel processing across a population of neurons.
					\item Multiple neurons are used to represent complex concepts. For example: the representation of a person may have a neuron for their name, a neuron for their profession, a neuron for their cat's name, and so on.
					\item The \underline{pattern of activation} of the neurons represent a symbolic concept.
					\item Semantic knowledge and lexical knowledge for a particular symbolic concept have different activation patterns.
					\item Units in neural networks are connected by weights that are modified by learning (positive weight $\to$ activation, negative weight $\to$ inhibition).
				\end{itemize}
		
		\subsection{Major Assumptions of Approaches}
			The major assumption of these approaches is that research must be done in the lab. This is believed for two key reasons:
			\begin{itemize}
				\item We must uncover the basic processes underlying cognition in order to fully understand it.
				\item Processes are stable across situations, and can only be researched under controlled conditions (such as in a lab).
			\end{itemize}
		
		\subsection{Other Approaches}
			\subsubsection{The Evolutionary Approach}
				\begin{itemize}
					\item Mental processes are subject to natural selection.
					\item Cognition is based off our history, and special processes have developed over time.
				\end{itemize}
				
			\subsubsection{The Ecological Approach}
				\begin{itemize}
					\item Cognitive processes develop with culture and differ depending on the context and situation.
					\item Analyzes how humans behave in context-specific situations. As a result of this approach, natural observation must be used instead of lab research.
					\item People focus on the eyes of others, because they show attention, desires, and more. This behavior might differ depending on the context.
				\end{itemize}
	\section{The Brain} \lecture{January 15, 2013}
		\subsection{Dependent Measures of Behavior}
			Experiments in cognitive psychology focus on measuring outward behavior. Experiments are usually (but not quite always) measured using two metrics:
			\begin{itemize}
				\item \textbf{Accuracy}. Give the test subject some information and see how well they can hold onto it over time. Measure the percentage of accurate results from an experiment.
				\item \textbf{Response Time}. If one task takes longer than another, it's usually an indicator that the longer task requires more brain power. This measure is typically used when accuracy is virtually perfect, because events occur over time in the mind. Time is a much more sensitive measure than accuracy.
			\end{itemize}

			Time can be effectively used to determine the \textbf{decision time} for a particular task. For example, say you have two tasks:
			\begin{enumerate}
				\item Hit a button when you see a light.
				\item Hit a button when you see a green light, but not when you see a red light.
			\end{enumerate}

			The response times are measured for both of these tasks. The average response time for the simpler task (task 1) is 150 ms, and it's 250 ms for the more complex task. By subtracting task 1 from task 2, we get 250 ms - 150 ms = 100 ms, which is the decision time.
			\\ \\
			Computers have made experiments like these much easier to conduct than before. Computers enable the ability to gather much more precise response times.
			\subsubsection{Modified/Damaged Brains}
				Patients with brain damage and transcranial magnetic stimulation (TMS) are both useful in assessing the role of various regions of the brain in various cognitive abilities.
				\\ \\
				Transcranial magnetic stimulation (TMS) allows researchers to artificially simulate brain damage by sending repeated electric pulses to an area of the brain.
			\subsubsection{Human Brain Lesions}
				Brain damage or damage simulated by TMS can be useful in determining how the brain works. There are typically two approaches researchers take:
				\begin{itemize}
					\item \textbf{What function is supported by a given brain region?} Patients with similarly-damaged brains are examined for common cognitive deficits.
					\item \textbf{What brain regions support a given cognitive function?} People with realizable behavior deficits are examined for common brain damage. 
				\end{itemize}
			\subsubsection{Brain Processing}
				There is no direct means of knowing exactly what's happening in the mind. It can only be inferred or introspected. There are a couple of technologies that are commonly used to achieve this.
				\begin{itemize}
					\item \textbf{Functional magnetic resonance imaging (fMRI)}. fMRI specializes in providing a spatially-accurate view, however it is bad at timing brain activity. It works by indirectly measuring blood flow throughout the brain using magnets to distinguish between oxygenated and deoxygenated blood. fMRI machines are extraordinarily expensive -- they cost millions of dollars.
					\item \textbf{Electroencephalography (EEG)}. EEG specializes in providing a temporally-accurate view, however it is bad at spatial representation (opposite of fMRI). It works by measuring electrical activity on the scalp. It's much faster than fMRI because electrical signals are instantaneous. EEG machines are not nearly as expensive as fMRI machines -- EEG machines cost about \$100,000. EEGs are used for event-related potential (ERP), which measures the brain's response that is a direct result of some event. 
				\end{itemize}
		\subsection{Structure of the Brain}
			\subsubsection{Brain Planes}
				There are several different views of the brain.
				\begin{itemize}
					\item \textbf{Coronal Plane}. A vertical slice down the width of the brain.
					\item \textbf{Horizontal Plane} (also known as an \textbf{Axial} view).
					\item \textbf{Sagittal Plane}. A vertical slice down the length of the brain.
					\item \textbf{Midsagittal Plane}. A slice down the very middle of the brain.
				\end{itemize}
				Take a look at the course slides (chapter 2, slide 9) for illustrations.
			\subsubsection{Brain Regions}
			There are many important brain regions, some of which concern cognitive psychology more than others. Most of the areas of interest for cognitive psychology lie in the \textbf{Forebrain} (the top portion of the brain). Some important regions are:
				\begin{itemize}
					\item \textbf{Cerebral Cortex}.
					\item \textbf{Medulla oblongata}.  It regulates life support, and is the main transmission point between the body and the brain.
					\item \textbf{Pons}. A neural relay, for the left body / right brain (yes, they're opposites!).
					\item \textbf{Cerebellum}. Coordinates muscular activity, handles general motor behavior and balance.
					\item \textbf{Corpus callosum}. Located at the center of the brain, it is where information is transmitted between brain hemispheres.
					\item \textbf{Amygdala}. Responsible for encoding and retrieving the emotional aspects of an experience. Note that you can never encode and retrieve information in a vacuum -- context is always present.
					\item \textbf{Hippocampus}. Acts as a key structure for memory (indexes memory). The hippocampus is to your memories as Google is to the web. It points you in the direction of what you're looking for.
					\item \textbf{Thalamus}. Serves as the a major relay centre.
				\end{itemize}

				Subcordical regions vary less from individual to individual. They're much more specialized at particular tasks than other regions.
			\subsubsection{Lobes of the Brain}
				There are four main lobes in the brain:
				\begin{itemize}
					\item \textbf{Frontal lobe} (at the front of the brain), responsible for complex cognitive activities such as decision-making, speech, motor activity, and executive functioning.
					\item \textbf{Parietal lobe} (at the top of the brain, but behind the frontal lobe), responsible for spational operations, mathematics, and sensations.
					\item \textbf{Temporal lobe} (below the frontal and parietal lobes), responsible for memory, recognition, and auditory abilities.
					\item \textbf{Occipital lobe} (at the very back of the brain), responsible for visual perception.
				\end{itemize}
		\subsection{Localization of Function}
			There is a lot history in cognitive psychology of claiming that functions are localized within the brain.
			\\ \\
			One theory was that different mental abilities were independent and autonomous functions that were carried out in different parts of the brain. The physical size of the regions was said to be related to the important of that particular cognitive function. They even made career suggestions based on bumps on your head, because bumps were said to indicate brain regions that were abnormally large. Obviously, this is not true, but localization might be! 
			\subsubsection{Faulty Assumptions}
				\begin{itemize}
					\item Brain location/region size $\ne$ power/capacity. Sometimes there is a loose relation between a brain region's size and its power, but it's not as black-and-white as it was described to be.
					\item Functions are not independent. This is especially false in the cortex because the whole brain is involved at some level with all tasks, only some regions are more active than others.
				\end{itemize}

			\subsubsection{Localization of Language (Double Dissociations)}
				Let's say you have two patients.
				\begin{itemize}
					\item A patient with damage to area $X$ is impaired for cognition $A$ but not for cognition $B$.
					\item A patient with damage to area $Y$ is impaired for cognition $B$ but not for cognition $A$.
				\end{itemize}

				This is a \textbf{double dissociation}, where two patients' deficits correspond to the opposite deficit as each other.
		
		\subsection{Brain Imaging Techniques}
			Brain imaging is mostly done with fMRI today. Brain imaging today can measure healthy, normal brains, whereas in the past there was only an opportunity to analyze damaged brains.
			\subsubsection{Functional Neuroimaging}
				Electrical activity on the scalp can be measured. \textbf{Event-related potential} (ERP) is a derivative of EEG. We also use PET (Positron Emission Tomography) and BOLD (blood oxygenated level dependent fMRI) to analyze metabolism. PET scans are typically avoided because they involve injecting radioactive substances into your blood.
				\\ \\
				In analyzing the results from these scans, the same \textbf{subtractive logic} is used, as was used in determining decision time. We feed two stimuli to the subject, and subtract the more simplistic stimuli's results from the more complex stimuli's results, in order to get the delta.
				\\ \\
				Donders in 1868 foresaw BOLD fMRIs, despite the fact that technology was nowhere near that advanced back then.
			\subsubsection{Cerebral Blood Supply}
				Blood flow increases with neuron activity. Functional MRIs measure the difference in the magnetic properties of oxygenated and deoxygenated blood (BOLD). BOLD is slow because metabolic processes are slow. BOLD takes several seconds to take a measurement, which is relatively long.
				\\ \\
				fMRI machines can be setup with goggles for the subject or a prism to look through (at a screen) in order to show them stimuli for use in experiments. The same experiments are possible inside an fMRI -- the only difference is that the subject is lying in a large magnetic tube.
				\\ \\
				The level of brain activity in various regions is illustrated on results by a color scheme. The more yellow a region is, the more active it is. Red regions are less active. Results and statistics are often superimposed over an MRI image. Statistics could also be plotted, withe the \% change being plotted against time.
	\section{Perception} \lecture{January 17, 2013}
		\textbf{Housekeeping Note}: the first exam is next Thursday (January 24, 2013), and it will contain 30 multiple-choice questions and 2 short answer questions.
		
		\begin{itemize}
			\item What our mind interprets is not a true representation of the world, but it's fairly close.
			\item Our retinas are magical. They define edges and boundaries for the objects we see.
			\item Fun fact: the retinal image is flipped left/right and upside down.
			\item The brain fills in many gaps, and makes many assumptions. It simplifies our visual and auditory world, and helps to make sense of it.
			\item The beauty of our cognitive architecture is we aren't actively aware of it.
			\item The way we perceive size and other qualities depends on our past experiences with similar stimuli. Some inferences our brain makes based on past experiences can sometimes lead us astray.
			\item Let's say you're looking at a book. The physical book is the \textbf{distal stimulus}, your retinal image of the book is the \textbf{proximal stimulus}, and your recognition of the object as a book is the \textbf{percept}.
			\item The distal stimulus and percept are not precise copies of one another.
			\item Perceived size, luminance, and color do not necessarily correspond in any simple way to the stimulus. Perception is not completely determined by the stimulus itself, it requires the perceiver's active participation.
		\end{itemize}

		\subsection{Bottom-Up Processes}
			\begin{itemize}
				\item Simple low-level functions that analyze for basic features, driven by stimuli.
				\item Often referred to as \textbf{data-driven} or \textbf{stimulus-driven}.
				\item Percepts are built from low-level features only.
				\item There are three general classes of bottom-up processes: template matching, feature analysis, and prototype matching.
			\end{itemize}
			
			\subsubsection{Template Matching}
				\begin{itemize}
					\item An external stimulus is matched with an internal template (a stored pattern in memory).
					\item A pattern is compared to all templates in memory and identified by the template that best matches it.
					\item Works well in machine vision applications, such as parsing the bank account/routing numbers on the bottom of cheques or the UPC code from a barcode.
				\end{itemize}

				There are a few problems with template matching, however.
				\begin{itemize}
					\item It needs lots of templates. You'd need an infinitely large cranium.
					\item It does not explain how we recognize new objects, or create new templates in the first place.
					\item It does not work well with surface variation of the stimuli (i.e. stimuli that are similar but not similar on the surface \textendash{} such as varying writing styles).
				\end{itemize}

			\subsubsection{Feature Analysis}
				\begin{itemize}
					\item Objects are composed of a combination of \underline{features}.
					\item Features are small/local templates that can be combined in many ways.
					\item The mind first recognizes individual features, then it recognizes the combination of them at a deeper level.
					\item \underline{Examples}: pandemonium, recognition by components (RBC; involves complex features), and word reading.
				\end{itemize}
				
				There are several advantages to feature analysis.
				\begin{itemize}
					\item It's more flexible than template matching.
					\item It doesn't require nearly as many templates as in template matching (certainly not an infinite number of templates).
				\end{itemize}

				There is lots of evidence supporting feature analysis.
				\begin{itemize}
					\item \textbf{Visual search}. If you have a box filled with letters, most of which are curved letters, and you're asked to find the X in the box, it'll jump out quickly. If you're trying to find a target object within a group of objects, if the target object has the same basic features as the rest of the group, the search must happen \underline{serially} because the visual cortex can't process the question. The reaction time remains constant as the set size increases if parallel (automatic) processing is occurring, otherwise it grows linearly.
					\item \textbf{Cortical feature detectors}. Recall these from the lecture on history.
				\end{itemize}
			\subsubsection{Prototype Matching}
				\begin{itemize}
					\item Matches a pattern to a stored, fuzzy representation called a \textbf{prototype}.
					\item A prototype is an idealized representation of a class of objects.
					\item Exact matching is not required.
					\item When you pass the featural level, the representation is compared to a prototype.
					\item The prototype for a class of objects differs from person to person based on their past experiences.
					\item Back in the day, Palm Pilots used prototype matching to allow people to type letters by writing them out. They used fuzzy matching to match peoples' scribbles with the letters of the alphabet.
				\end{itemize}
		\subsection{Top-Down Processes}
			\begin{itemize}
				\item Often referred to as \textbf{theory-driven} or \textbf{conceptually-driven} processing.
				\item Knowledge, theories, and expectations influence perception based on past experiences.
				\item \underline{Examples}: context effect, and change blindness.
			\end{itemize}

			\subsubsection{Context Effect}
				\begin{itemize}
					\item Colors are influenced by their surrounding colors.
					\item Dale Purves discovered that the way we perceive objects is determined by our historical success or failure with similar objects. This happens at the neural level as well.
					\item A basic structure exists for the initial recognition, prior to any relevant experiences having occurred. This is the base case.
					\item Lots of pruning occurs over time.
					\item The gist of the context effect: context in which objects appear affects perception.
					\item See \href{http://www.purveslab.net/seeforyourself/}{Dale Purves' website, Purves Lab} (read that out loud), for a demonstration of the context effect in action. In particular, take a look at the ``Brightness contrast with color: cube'' demo.
				\end{itemize}

			\subsubsection{Change Blindness}
				\begin{itemize}
					\item You won't perceive changes that don't change the meaning of what you're perceiving.
					\item \underline{Flicker paradigm}: you need a break for the perceptual system to reset. You can't just go from the original image to a modified image (you'll notice the difference) \textendash{} you need a flicker between the images in order to not notice the difference.
					\item We perceive objects for the gist of it. We don't perceive all of the little, minor details.
				\end{itemize}
		\subsection{Other Views of Perception} \lecture{January 22, 2013}
			\subsubsection{Gestalt Approaches}
				\begin{itemize}
					\item Has a bit of a top-down feel to it.
					\item Focuses on understanding how we come to recognize objects as forms.
					\item Form perception: segregation of displays into objects and background (vase vs. two faces illusion).
					\item It's a holistic view \textendash{} how we impose structure on objects.
					\item It's very difficult to view the two faces and the vase in the illusion at the same time. You can force yourself to view one or the other, but seeing both at once isn't easy.
					\item The goal is to derive Gestalt principles of perceptual organization.
					\item \underline{Law of Pragnanz}: we tend to select the organization that yields the simplest and most stable shape or form, out of all possible interpretations.
					\item There are various cues for simplification, including:
						\begin{itemize}
							\item No grouping.
							\item Proximity (several groups of 2, for example).
							\item Similarity of color.
							\item Similarity of size.
							\item Similarity of orientation.
							\item Common fate (objects that move together).
							\item Symmetry.
							\item Parallelism.
							\item Continuity.
							\item Closure.
						\end{itemize}
					\item Once you perceive something in a certain way, it alters your perception of other things.
					\item Combining cues can help you figure out what to focus on and what's considered to be less focus-worthy (the background).
					\item Our perceptual system groups and categorizes objects. Our memory system is organized hierarchically and categorically.
					\item The human perceptual system is very efficient at processing faces. It's very good at recognizing, perceiving, and remembering faces. You can also read a lot (including sincerity and intentions) from faces.
					\item We impose facial structure on things that aren't faces (i.e. dents on Mars have some face-like features). Once you get the perception of a face, the other details fill themselves in \textendash{} it's a very holistic, top-down, and automatic process.
					\item It takes longer to parse faces upside down because it is not how we naturally view faces \textendash{} all of a sudden, we must apply feature analysis. The process is more serial, focusing on details.
				\end{itemize}

			\subsubsection{Neuropsychology}
				\begin{itemize}
					\item Patients with brain damage are often studied.
					\item The emphasis of these studies is on the preserved cognitive abilities and the deficits.
					\item Helps to identify which brain regions are special purpose devices that have a particular function, such as face recognition.
				\end{itemize}

			\subsubsection{Visual Agnosia}
				\begin{itemize}
					\item \textbf{Visual agnosia} is the inability to identify certain objects by sight.
					\item There are two types of visual agnosia:
						\begin{enumerate}
							\item \textbf{Apperceptive agnosia}, which is the inability to form stable (presemantic) representations of objects. This is a low-level processing deficit. The retina is okay, but there is damage to one side of the brain near the back of the head.
							\item \textbf{Associative agnosia}, which is where percepts can be formed but cannot be identified (cannot achieve a correct semantic description). This type of agnosia is category-specific. This is a high-level cognitive deficit.
						\end{enumerate}
					\item \textbf{Prosopagnosia} is a type of associative agnosia where faces cannot be recognized.
					\item No faces can be recognized, not even those of family members. Not even their own.
					\item Cognitive processes that we're really good at, like face recognition, have a higher risk of damage because more cordical space is dedicated to them.
					\item People with prosopagnosia use contextual cues (such as clothes, etc.) to identify people instead of being able to identify people based on their face.
					\item Working with inverted faces is easier for people with prosopagnosia because their detail/feature analysis processes are more tuned and efficient than those of people who do not have prosopagnosia.
					\item There is evidence that there is still covert, unconscious recognition of these faces in people with prosopagnosia, however. The unconscious feeling of familiarity is called a GSR (a \textbf{Galvonic Skin Response}).
					\item You might have a similar experience with a GSR when you encounter someone out of their usual context.
					\item There is dedicated neural tissue in the brain for face processing, called the Fusiform cortex, also known as the \textbf{Fusiform Face Area} (FFA). This neural tissue is located in the temporal lobe of the brain.
					\item We tend to focus on the eyes, but people with prosopagnosia focus on less diagnostic features like mouths, etc. There's a tendency for them to use less distinguishable features to discriminate, which obviously isn't terribly effective.
					\item Why is this information stored in the temporal lobe? So it can be used in close proximity to other functions.
				\end{itemize}
			\subsubsection{Capgras Syndrome}
				\begin{itemize}
					\item \textbf{Capgras Syndrome} is the flip of prosopagnosia: you can recognize the person but you don't get the unconscious sense of familiarity.
					\item Everyone feels like an imposter, even people who are very close.
					\item There was a woman who suffered from Capgras Syndrome who claimed to have over 80 husbands because they all felt like imposters who had been replaced.
					\item People with Capgras Syndrome are not prosopagnosic. They just don't have any GSRs occurring.
					\item They have a delusional feel.
					\item One explanation is that it results from an attempt to reconcile the lack of GSR response with the fact that they know the person looks identical to someone they know.
					\item Some people believe the Fusiform area is actually reserved for our most advanced area of expertise, which just happens to be face processing.
					\item Prosopagnosia is where you can't sense overtly, but you can sense covertly. Capgras Syndrome is the opposite: you can sense overtly, but not covertly.
				\end{itemize}
\end{document}
