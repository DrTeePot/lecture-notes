\documentclass[]{article}
\usepackage[margin = 1.5in]{geometry}
\setlength{\parindent}{0in}
\usepackage{amsfonts}
\usepackage{amssymb}
\usepackage{hyperref}
\usepackage[T1]{fontenc}
\usepackage{ae,aecompl}

\begin{document}

	\title{\bf{PSYCH 207: Cognitive Processes}}
	\date{Winter 2013, University of Waterloo \\ \center Notes written from Jonathan Fugelsang's lectures.}
	\author{Chris Thomson}
	\maketitle
	\newpage

	% January 8, 2013
	\section{Introduction \& Course Structure}
		\subsection{Course Structure}
			The grading scheme is four in-class non-cumulative multiple-choice exams, equally weighted. There is also a 4\% bonus for research participation through SONA. You should get the textbook.
			\\ \\
			See the course syllabus for more information -- it's available on \href{https://learn.uwaterloo.ca/}{Waterloo LEARN}.
		
		\subsection{Introduction to Cognitive Processes}
			\begin{quote}
				``Cognitive psychology refers to all processes by which the sensory input is transformed, reduced, elaborated, stored, recovered, and used." \textendash{}  Neisser, 1967
			\end{quote}
			\textbf{Cognitive psychology} involves perception, attention, memory, knowledge, reasoning, and decision making.
			\\ \\
			\textbf{Cognitive processes} are everything that goes on in our mind that affects our environment. Many of these processes are completely unconscious.
			\\ \\
			Conscious experience is an \underline{active reconstructive process}. The external world and our internal representation of that world is \emph{not} an exact match. Our brain ends up filling in many gaps, making many assumptions.
			\\ \\
			Our brain cannot decontextualize the world.
\end{document}