\documentclass[]{article}
\usepackage[margin = 1.5in]{geometry}
\setlength{\parindent}{0in}
\usepackage{amsmath}
\usepackage{amsthm}
\usepackage{amsfonts}
\usepackage{amssymb}
\usepackage{hyperref}
\usepackage{wasysym}
\usepackage{soul}
\usepackage{cleveref}
\usepackage[T1]{fontenc}
\usepackage{ae,aecompl}

\newtheorem{theorem}{Theorem}[section]
\newtheorem{proposition}{Proposition}[section]

\theoremstyle{definition}
\newtheorem{problem}{Problem}[section]
\newtheorem{defn}{Definition}[section]
\newtheorem{ex}{Example}[section]

\newcommand{\union}{\cup}
\newcommand{\intersection}{\cap}

\setlength{\marginparwidth}{1.5in}
\newcommand{\lecture}[1]{\marginpar{{\footnotesize $\leftarrow$ \underline{#1}}}}

\begin{document}
	\let\ref\Cref
	\title{\bf{MATH 239: Introduction to Combinatorics}}
	\date{Winter 2013, University of Waterloo \\ \center Notes written from Bertrand Guenin's lectures.}
	\author{Chris Thomson}
	\maketitle
	\newpage
	
	\section{Introduction, Permutations, and Combinations} \lecture{January 7, 2013}
		\subsection{Course Structure}
			The grading scheme is \st{50\%} 55\% final, 30\% midterm, \st{10\% quizzes} 5\% participation marks (clicker questions), and 10\% assignments. There are \st{ten quizzes and} ten assignments. \st{The quizzes will not be announced in advance, and each quiz consists of a single clicker question}. Assignments are typically due on Friday mornings at 10 am. in the dropboxes outside of MC 4066. The midterm exam is scheduled for March 7, 2013, from 4:30 pm - 6:20 pm. There is no textbook for the course, but there are course notes available at Media.doc in MC, and they're \emph{highly} recommended.
			\\ \\
			MATH 239 is split into two parts: counting (weeks 1-5) and graph theory (weeks 6-12).
			\\ \\
			See the course syllabus for more information -- it's available on \href{https://learn.uwaterloo.ca/}{Waterloo LEARN}.

		\subsection{Sample Counting Problems}
			\begin{problem}
				How many ways are there to cut a string of length 5 into parts of sizes 1 and 2?
			\end{problem}
		
			Here are a few example cuts:
			\\ \\
			\begin{tabular}{|c|c|c|c|c|}
				\hline
				1 & 2 & 3 & 4 & 5 \\
				\hline
			\end{tabular}
			5 cuts: size 1, size 1, size 1, size 1, size 1.

			\begin{tabular}{|c|c|c|c|c|}
				\hline
				1 & 1 & 2 & 2 & 3 \\
				\hline
			\end{tabular}
			3 cuts: size 2, size 2, size 1.
		
			\begin{tabular}{|c|c|c|c|c|}
				\hline
				1 & 2 & 2 & 3 & 3 \\
				\hline
			\end{tabular}
			3 cuts: size 1, size 2, size 2.
			\\ \\
			This is a finite problem. You could count all of the possibilities manually in this case. However, this problem could be made more complicated to a point where manually counting all possibilities would become quite cumbersome, as is the case in the next problem.
		
			\begin{problem}[Cuts]
				How many ways are there to cut a string of size 372,694 into parts of sizes 3, 17, 24, and 96?
			\end{problem}
		
			\begin{defn}
				A positive integer $n$ has a \textbf{composition} $(m_1, m_2, \ldots, m_k)$, where $m_1, \ldots, m_k$ are positive integers and where $n = m_1 + m_2 + \ldots + m_k$. $m_1, \ldots, m_k$ are the \textbf{parts} of the composition.
			\end{defn}
		
			Problem 1 could be rephrased as looking for the number of compositions of 5 where all parts are 1 or 2.
		
			\begin{problem}
				How many compositions of $n$ exist such that all parts are odd?
			\end{problem}
		
			\begin{problem}[Binary Strings]
				Let $S = a_1, a_2, \ldots, a_n$ where $a_i \in \{0, 1\}$. How many strings $S$ exist?
			\end{problem}
		
			For each $a_i$, there is the choice between $0$ or $1$, and that choice is independent for each character of the string (for each $a_i$). So, there are $2^n$ binary strings of length $n$.
		
			\begin{problem}
				How many binary strings of size $n$ exist that do not include the substring $1100$?
			\end{problem}
		
			For example: $1010\underline{1100}101 \not \in S$.
		
			\begin{problem}
				How many binary strings of size $n$ exist such that there is no odd-length sequences of zeroes? 
			\end{problem}
		
			For example: $1001001\underline{000}11 \not \in S$.
		
			\begin{problem}[Recurrences]
				How many times does a recursive function get called for a particular input $n$?
			\end{problem}
		
		\subsection{Sample Graph Theory Problems}
			\begin{problem}
				For any arbitrary map of regions, color the regions such that no two touching boundaries do not have the same color, with the least number of colors possible.
			\end{problem}
		
			The \textbf{four-color theorem} (proven later in the course) states that you can always do this with four colors. It's also always possible to color these regions with five colors. It's \emph{sometimes} possible to color the regions with three or fewer colors, depending on the layout of the regions and their boundaries.
		
		\subsection{Permutations and Combinations}
			\subsubsection{Set Notation}
				The usual set and sequence notation is used in this course. $(1, 2, 3)$ is a sequence (where order matters), and $\{1, 2, 3\}$ is a set (where order does not matter).
				\\ \\
				We will also be using one piece of notation you may not be familiar with: $[n] := \{1, 2, \ldots, n\}$.

			\subsubsection{Permutations}
				\begin{defn}
					A \textbf{permutation} of $[n]$ is a rearrangement of the elements of $[n]$. The number of permutations of a set of $n$ objects is $n \times (n - 1) \times \ldots \times 1 = n!$. 
				\end{defn}
			
				For example: the number of permutations of $6$ objects is $6 \times 5 \times 4 \times 3 \times 2 \times 1 = 6!$ permutations.
				\\ \\
				Why is this the case? Simple: there are $n$ choices for the first position, $(n - 1)$ choices for the second position, $(n - 2)$ choices for the third position, and so on, until there's $1$ choice for the $n$th position.

				\begin{defn}
					A \textbf{$k$-subset} is a subset of size $k$.
				\end{defn}

				\begin{problem}
					How many $k$-subsets of $[n]$ exist?
				\end{problem}

				Let's consider a more specific case: how many 4-subsets of 6 are there? $\frac{6 \times 5 \times 4 \times 3}{4!}$.
				\\ \\
				For simplicity's sake, we will introduce notation for this, which we will refer to as a \textbf{combination}, denoted as 
					${n \choose k} = \frac{n!}{k!(n-k)!}$ where $n, k \in \mathbb{Z} \ge 0$.
				
				\begin{proposition}
					There are ${n \choose k}$ $k$-subsets of $[n]$.
				\end{proposition}
				
				\subsubsection{Application: Binomial Theorem} \lecture{January 9, 2013}
					\begin{align*}
						(1 + x)^n = \sum_{k = 0}^{n} {n \choose k} x^k
					\end{align*}
					
					Why is this true?
					
					\begin{align*}
						(1 + x)^3 = \overbrace{(1 + x)}^1\overbrace{(1 + x)}^2\overbrace{(1 + x)}^3 = 1 + 3x + 3x^2 + x^3 = {3 \choose 0} + {3 \choose 1}x + {3 \choose 2}x^2 + {3 \choose 3}x^3
					\end{align*}
					
					\begin{proof}
						\begin{align*}
							(1 + x)^n = \overbrace{(1 + x)}^1 \overbrace{(1 + x)}^2 \cdots \overbrace{(1 + x)}^n
						\end{align*}
						
						In order to get $x^k$, we need to choose $x$ in $k$ of $\{1, \ldots, n \}$. There are ${n \choose k}$ ways of doing this.
					\end{proof}
		
		\section{Simple Tools for Counting}
			\subsection{Partitioning}
				Sets $S_1, S_2$ partition the set $S$ if $S = S_1 \union S_2$ and $S_1 \intersection S_2 = \emptyset$.
				
				\begin{ex}
					\begin{align*}
						S &= [5] = \begin{cases} 
							S_1 = \{1, 2 \} \\
							S_2 = \{3, 4, 5\}
						\end{cases} \\
						|S| &= |S_1| + |S_2|
					\end{align*}
				\end{ex}
				
				\begin{proposition}
					$\displaystyle 2^n = \sum_{k = 0}^{n} {n \choose k}$
				\end{proposition}
				
				\begin{proof}
					We will discuss two proof methods.
					
					\begin{enumerate}
						\item \textbf{Algebraic proof}. Set $x = 1$ in the Binomial Theorem.
						\item \textbf{Combinatorial proof}. We will count the left-hand side and the right-hand side in different ways to reach the same result.
						
						Let $S$ be the set of subsets of $[n]$. $|S| = 2^n$, since for every element of $[n]$ we have two possibilities: include or don't include the element in $S$.
						
						\underline{Aside}: suppose $n = 2$. Then $S = \{\emptyset, \{1\}, \{2\}, \{1, 2\}\}$. 
						
						Partition $S$ into $S_0, S_1, \ldots, S_n$, where $S_k$ is the set of $k$-subsets of $[n]$.
						
						\begin{align*}
							\underbrace{|S|}_{2^n} = \underbrace{|S_0|}_{{n \choose 0}} + \underbrace{|S_1|}_{n \choose 1} + \cdots + \underbrace{|S_n|}_{n \choose n}
						\end{align*}
					\end{enumerate}
				\end{proof}
				
				\begin{proposition}
					$\displaystyle {n \choose k} = {n - 1 \choose k} + {n - 1 \choose k - 1}$
				\end{proposition}
				
				\begin{proof}
					Let $S$ be the set of $k$-subsets of $[n]$. Then $|S| = {n \choose k}$. Partitioning $S$, let $S_1$ be the subsets of $S$ containing $n$, and let $S_2$ be the subsets of $S$ not containing $n$.
					\\ \\
					It's easy to see that $|S_1| = {n - 1 \choose k - 1}$ ($n$ is already included in our choices) and $|S_2| = {n - 1 \choose k}$. We now have $|S| = |S_1| + |S_2| = {n - 1 \choose k - 1} + {n - 1 \choose k}$.
				\end{proof}
				
			\subsection{Pascal's Triangle}
				Pascal's Triangle is a triangle where each value is determined by the sum of its two direct parents. The uppermost value is 1.
				\\ \\
				\begin{center}
					\begin{tabular}{rccccccccc}
						$n = 0$:&    &    &    &    &  1\\\noalign{\smallskip\smallskip}
						$n = 1$:&    &    &    &  1 &    &  1\\\noalign{\smallskip\smallskip}
						$n = 2$:&    &    &  1 &    &  2 &    &  1\\\noalign{\smallskip\smallskip}
						$n = 3$:&    &  1 &    &  3 &    &  3 &    &  1\\\noalign{\smallskip\smallskip}
						$n = 4$:&  1 &    &  4 &    &  6 &    &  4 &    &  1\\\noalign{\smallskip\smallskip}
					\end{tabular}
				\end{center}
				
				\begin{proposition}
					$\displaystyle {q + r \choose q} = \sum_{i = 0}^{r} {q + i - 1 \choose q - 1}$
				\end{proposition}
				
				For example: let $q = 3, r = 2$. Then we have: $\displaystyle {5 \choose 3} = {2 \choose 2} + {3 \choose 2} + {4 \choose 2}$.
				
				\begin{proof}
					Let $S$ be the set of $q$-subsets of $[q + r]$, so $|S| = {q + r \choose q}$. Partition $S$ such that $S_i$ is the set of $q$-subsets where the largest element is $q + i$ (where $i = 0, \ldots, r$).
					\\ \\
					We have: $|S| = |S_0| + |S_1| + \cdots + |S_r|$. Note that $|S_i| = {q + i - 1 \choose q - 1}$. That gives us:
					
					\begin{align*}
						\underbrace{|S|}_{{q + r \choose q}} = \underbrace{\sum_{i = 0}^{r} |S_i|}_{{q + i - 1 \choose q - 1}}
					\end{align*}
				\end{proof}
		
		\subsection{Injections, Bijections, and Onto} \lecture{January 11, 2013}
			Let's consider a function $f: S \to T$.
			\\ \\
			The function $f$ is an \textbf{injection} if for all $x_1, x_2 \in S, x_1 \ne x_2$ such that $f(x_1) \ne f(x_2)$. That is, no element in the codomain $T$ is the image of more than one element in $S$.
			\\ \\
			The function $f$ is \textbf{onto} (or a \textbf{surjection}) if for all $y \in T$, there exists $x \in S$ such that $f(x) = y$. That is, all elements in the codomain $T$ are the image of an element in $S$. Multiple elements in $S$ can map to the same element in $T$.
			\\ \\
			The function $f$ is a \textbf{bijection} if it is both injective and onto. That is, there is a one-to-one mapping between elements in $S$ and $T$, and vice versa.
			
			\begin{proposition}
				If $S$ and $T$ are finite, $|S| = |T|$ if $f$ is bijective.
			\end{proposition}
			
			\begin{defn}
				$g$ is the inverse of $f$ if:
				\begin{enumerate}
					\item For all $x \in S, g(f(x)) = x$.
					\item For all $y \in T, f(g(y)) = y$.
				\end{enumerate}
			\end{defn}
			
			In order to show that a function has a bijection, find the inverse function.
			
			\subsection{Application: Binomial Coefficients}
				\begin{proposition}
					$\displaystyle {n \choose k} = {n \choose n - k}$
				\end{proposition}
				
				\begin{proof}
					I will prove this proposition in a combinatorial using the bijection technique. We want to show that the cardinalities are the same. 
					\\ \\
					Let $S_1$ be the set of $k$-subsets of $[n]$, so $|S_1| = {n \choose k}$, as shown earlier. Let $S_2$ be the set of $(n-k)$-subsets of $[n]$, so $|S_2| = {n \choose n - k}$. We need to show that $|S_1| = |S_2|$, so we need to show that there is a bijection between $S_1$ and $S_2$, $f: S_1 \to S_2$.
					\\ \\
					\underline{Aside}: suppose $n = 5$ and $k = 2$. Then, the bijection could be $\{1, 3\} \to \{2, 4, 5\}$ (the complement function).
					\\ \\
					The bijective function is $f(A) = [n]\backslash A$ (the complement function). Check: $f$ is its own inverse (let $g = f$).
				\end{proof}
				
	\section{Power Series and Generating Functions}
		\subsection{Power Series}
			\begin{defn}
					Let $(a_0, a_1, \ldots)$ be a sequence of rational numbers. Then:
					\begin{align*}
						A(x) = \sum_{i \ge 0} a_i x^i
					\end{align*}
					
					This is called a \textbf{power series}.
			\end{defn}
			
			\begin{ex}
				$a_i = 2^i \implies A(x) = 1 + 2x + 4x^2 + 8x^3 + \cdots$
			\end{ex}

			If the sequence is finite, it is just a polynomial.
			
			\subsubsection{A General Counting Problem}
				Let $S$ be the the set of objects $\sigma$. Each object $\sigma$ has a weight $w(\sigma)$. We want to know how many objects of $S$ have some weight $k$.
				
				\begin{ex}
					Let $\sigma \subseteq [n], w(\sigma) = |\sigma|$. The question becomes: how many $k$-subsets of $[n]$?
				\end{ex}
				
				\begin{ex}
					Let $\sigma$ be the set of coins (1\cent, 5\cent, 10\cent, 25\cent, \$1, \$2), and let $w(\sigma)$ be the total value of the coins in $\sigma$.
					\\ \\
					The question becomes: how many sets of coins have total value $k$? In other words, how many ways are there to give change on $k$ amount?
				\end{ex}
		\subsection{Generating Functions}
			\begin{defn}
				Given $S$ as the set of objects $\sigma$ and a weight function $w(\sigma)$:
				\begin{align*}
					\phi_S(x) = \sum_{\sigma \in S} x^{w(\sigma)}
				\end{align*}
				
				This is called the \textbf{generating function for S, w}.
			\end{defn}
			
			\begin{ex}
				Let $S = \{ \sigma | \sigma \subseteq \{1, 2, 3\} \}, w(\sigma) = |\sigma|$.
				
				\begin{center}
					\begin{tabular}{c|c|c}
						$\sigma$ & $w(\sigma)$ & $x^{w(\sigma)}$ \\ \hline
						$\emptyset$ & 0 & 1 \\
						\{1\} & 1 & x \\
						\{2\} & 1 & x \\
						\{3\} & 1 & x \\
						\{1, 2\} & 2 & $x^2$ \\
						\{1, 3\} & 2 & $x^2$ \\
						\{2, 3\} & 2 & $x^2$ \\
						\{1, 2, 3\} & 3 & $x^3$ \\
					\end{tabular}
				\end{center}
				
				$\phi_S(x) = 1 + 3x + 3x^2 + x^3 = (1 + x)^3$ is the generating function. Notice the coefficients are the number of objects whose weight is equal to the exponent.
			\end{ex}
			
			\textbf{Remember}: the generating function for $S$ with weights $w$ is $\phi_S(x) = \sum_{k \ge 0} a_k x^k$, where $a_k$ is the number of objects of size $k$ in $S$.

			\begin{ex}
				\lecture{January 14, 2013}
				Let $S$ be the set of subsets of [n], and $w(\sigma) = |\sigma|$.
				\begin{align*}
					\phi_S(x) = \sum_{k \ge 0} {n \choose k} x^k = (1 + x)^n
				\end{align*}
				
				Note that ${n \choose k}$ is included because that's the number of $k$-subsets of $[n]$. $\phi_S(x) = (1 + x)^n$ by the binomial theorem.
			\end{ex}

			\begin{proposition}
				Let $\phi_S(x)$ be the generating function for finite-size $S$ with weight $w$. Then:
				\begin{enumerate}
					\item $\phi_S(1) = |S|$
					\item $\phi_S'(1) = $ sum of the weight of all the objects in $S$.
				\end{enumerate}

				Together, we get:
				\begin{align*}
					\frac{\phi_S'(1)}{\phi_S(1)} = \text{average weight of objects in } S
				\end{align*}
			\end{proposition}

			Considering the previous example again, we know that $|S| = 2^n$ and the average weight is clearly $\frac{n}{2}$. We can verify that with this proposition.
			\begin{align*}
				\phi_S(x) = (1 + x)^n &\implies \phi_S(1) = (1 + 1)^n = 2^n \\
				\phi_S'(x) = n(1 + x)^{n - 1} &\implies \phi_S'(1) = n2^{n - 1} \\
				\text{average weight } &= \frac{\phi_S'(1)}{\phi_S(1)} = \frac{n2^{n - 1}}{2^n} = \frac{n}{2}
			\end{align*}

			We'll now prove the proposition more generally.
			\begin{proof}
				We will prove the two parts of the proposition separately. The average weight clearly follows from those two results.
				\begin{enumerate}
					\item 
						\begin{align*}
							\phi_S(x) = \sum_{\sigma \in S} x^{w(\sigma)}
						\end{align*}
						Choose $x = 1$. Then $\phi_S(1) = \sum_{\sigma \in S} 1 = |S|$.

					\item 
						\begin{align*}
							\phi_S'(x) = \sum_{\sigma \in S} w(\sigma)x^{w(\sigma) - 1}
						\end{align*}
						Choose $x = 1$. Then $\phi_S'(1) = \sum_{\sigma \in S} w(\sigma) \cdot 1 = $ total weight of all objects in $S$.
				\end{enumerate}
			\end{proof}

			In order to work further with generating functions, we'll first need to learn how to manipulate power series generally.
		\subsection{Working With (Formal) Power Series}
			For the following definitions, we will assume the following power series are defined:
			\begin{align*}
				A(x) &= a_0 + a_1x + a_2x^2 + \cdots \\
				B(x) &= b_0 + b_1x + b_2x^2 + \cdots
			\end{align*}

			\begin{defn}
				We define \textbf{addition of power series} as follows.
				\begin{align*}
					A(x) + B(x) := \sum_{n \ge 0} (a_n + b_n)x^n
				\end{align*}

				Note that this definition is consistent with the definition of addition for polynomials.
			\end{defn}

			\begin{defn}
				We define \textbf{multiplication of power series} as follows.
				\begin{align*}
					A(x)B(x) := \sum_{n \ge 0} \sum_{k = 0}^{n} a_k b_{n - k}x^n
				\end{align*}
			\end{defn}

			\begin{ex}
				\begin{align*}
					&(1 + x + x^2 + x^3 + \cdots)(1 - x) \\
					&= 1 \cdot 1 + (1 \cdot -1 + 1 \cdot 1)x + (1 \cdot 1 + 1 \cdot -1)x^2 + \cdots \\
					& = 1
				\end{align*}
			\end{ex}

			\begin{defn}
				$B(x)$ is the \textbf{inverse} of $A(x)$ if $A(x)B(x) = 1$ (alternatively, $B(x)A(x) = 1$).
				\\ \\
				We will use the notation $B(x) = \frac{1}{A(x)}$ to indicate that $B(x)$ is the inverse of $A(x)$, and vice versa. 
			\end{defn}

			\begin{ex}
				The inverse of $(1 + x + x^2 + \cdots)$ is $(1 - x)$, as shown in the previous example.
			\end{ex}

			\textbf{Question}: does every power series have an inverse? No. For example, $(x + x^2)$ does not have an inverse. But suppose for a moment that it does.
			\begin{proof}
				If $(x + x^2)$ has an inverse, there would exist constants $b_i$ such that
				\begin{align*}
					(x + x^2)(b_0 + b_1x + b_2x^2 + b_3x^3 + \cdots) = 1
				\end{align*}
				Clearly, this is impossible. In order to equal 1, $b_0$ must multiply by some constant term, however there is no other constant term. Therefore, our assumption was incorrect, meaning $(x + x^2)$ does not have an inverse.
			\end{proof}

			\textbf{Remark}: if a power series does not have a constant term then it has no inverse, and vice versa.
			\begin{theorem}
				If the constant term of $A(x)$ is non-zero, then $A(x)$ has an inverse, and we can find it.
			\end{theorem}

			\textbf{Notation}: given a power series $A(x)$, we say $[x^k]A(x)$ represents the coefficient of $x^k$.

			\begin{ex}
				Find the inverse of $1 - x + x^2 - x^3 + x^4 - \cdots$.
				\begin{align*}
					\underbrace{(1 - x + x^2 - x^3 + x^4 - \cdots)(b_0 + b_1x + b_2x^2 + \cdots)}_{(\star)} = 1
				\end{align*}
				
				\begin{align*}
					[x^0](\star) = 1 \implies 1 \cdot b_0 = 1 &\implies b_0 = 1 \\
					[x^1](\star) = 0 \implies 1 \cdot b_1 - 1 \cdot b_0 = 0 &\implies b_1 = b_0 = 1 \\
					[x^2](\star) = 0 \implies 1 \cdot b_2 - 1 \cdot b_1 + 1 \cdot b_0 = 0 &\implies b_2 = b_1 - b_0 = 0
				\end{align*}
				Similarly, $b_3 = b_4 = \cdots = 0$. Thus, the inverse is $(1 + x)$.
			\end{ex}

			It's clear that the inverse for a power series is \textbf{unique}. We made no choices when determining the inverse, therefore it must be unique.
			\subsubsection{Finding Inverses} \lecture{January 16, 2013}
				Given:
				\begin{align*}
					A(x) = \sum_{n \ge 0} a_nx^n \text{ where } (a_0 \ne 0)
				\end{align*}
				We want to find:
				\begin{align*}
					B(x) = \sum_{n \ge 0} b_nx^n
				\end{align*}
				We'll start by finding $b_0$:
				\begin{align*}
					[x^0]A(x)B(x) = a_0b_0 = 1 \implies b_0 = \frac{1}{a_0}
				\end{align*}
				Notice that if we didn't have the restriction on $a_0$, we would've run into trouble here. Now, suppose you found $b_0, b_1, \ldots, b_{n - 1}$ for $n \ge 1$. Find $b_n$.
				\begin{align*}
					b_n = \frac{1}{a_0} \cdot (-a_1b_{n - 1} - a_2b_{n - 2} - \ldots - a_nb_0)
				\end{align*}

				\begin{proposition}
					Let $A(x)$ and $P(x)$ be formal power series.
					Suppose the constant for $A(x)$ is not 0. Then there exists a unique $B(x)$ such that $A(x)B(x) = P(x)$.
				\end{proposition}
				\textbf{Some useful formul\ae}:
					\begin{align*}
						\frac{1}{1 - x} = 1 + x + x^2 + x_3 + \cdots &\text{ since } 1 = (1 + x + x^2 + \cdots)(1 - x) \\
						\frac{1 - x^{k+1}}{1-x} = 1 + x + x^2 + \cdots + x^k &\text{ since } 1 - x^{k+1} = (1 + x + x^2 + \cdots + x^k)(1-x) 
					\end{align*}
			\subsubsection{Compositions}
				\begin{defn}
					Let $A(x)$ and $B(x)$ be formal power series defined by:
					\begin{align*}
						A(x) = \sum_{n \ge 0} a_n x^n \hspace{1cm} B(x) = \sum_{n \ge 0} b_n x^n
					\end{align*}
					We define the \textbf{composition} as:
					\begin{align*}
						A(B(x)) := \sum_{n \ge 0} a_n [B(x)]^n
					\end{align*}
				\end{defn}

				\begin{ex}
					Let $A(x) = 1 + x + x^2 + x^3 + \cdots$ and $B(x) = 2x$. Then the composition is $A(B(x)) = 1 + 2x + 4x^2 + 8x^3 + \cdots$.
				\end{ex}

				\textbf{Question}: is the composition of two formal power series a formal power series itself? No.
				\begin{ex}
					Suppose we pick $A(x) = 1 + x + x^2 + x^3 + \cdots$ and $B(x) = 1 + x$.
					\\ \\
					We have $A(B(x)) = 1 + (1 + x) + (1 + x)^2 + (1 + x)^3 + \cdots$. This cannot be a formal power series because a formal power series requires that all coefficients ($a_k$) need to be rational numbers, and in this case $a_0$ is infinite.
				\end{ex}

				\begin{ex}
					Suppose we pick $A(x) = 1 + x + x^2 + x^3 + \cdots$ and $B(x) = x + x^2$.
					\\ \\
					We have $A(B(x)) = 1 + (x + x^2)^1 + (x + x^2)^2 + \cdots$. In order to show that this is a formal power series, we need to show that the coefficient of $x^k$ (which is $a_k$) is finite for all $k$.
				\end{ex}
				
				\begin{theorem}
					Let $A(x)$ and $B(x)$ be formal power series, and $[x^0]B(x) = 0$ ($B(x)$'s constant is zero). Then $A(B(x))$ is a formal power series.
				\end{theorem}

				\underline{Aside}:
				\begin{align*}
					a(b(x)) = 1 + (\underbrace{x + x^2}_{x(1+x)})^1 + (\underbrace{x + x^2}_{x(1+x)})^2 + (\underbrace{x + x^2}_{x(1+x)})^3 + \cdots
				\end{align*}

				\begin{proof}
					We need to show that for any fixed $k$, $[x^k]A(B(x))$ is finite.
					\\ \\
					Let $R(x)$ be such that $B(x) = xR(x)$. Then we have:
					\begin{align*}
						[x^k]A(B(x)) &= [x^k] \sum_{n \ge 0} a_n[B(x)]^n \\
						&= [x^k] \sum_{n \ge 0} a_n x^n R(x) \\
						&= [x^k] \sum_{n \ge 0}^{k} a_n x^n (R(x))^n
					\end{align*}

					We made the last sum finite because we're interested in the coefficient of $x^k$ only. Note that this last line shows we can determine the coefficient in a finite number of steps, since $k$ is finite.
				\end{proof}

				\begin{ex}
					Let $y \mapsto x^2$ (I will call $x^2$, $y$).
					\begin{align*}
						\frac{1}{1 - x^2} = \frac{1}{1 - y} &= 1 + y + y^2 + y^3 + \cdots \\
					&= 1 + x^2 + x^4 + x^6 + \cdots
					\end{align*}

					We used composition with $A(x) = 1 + y + y^2 + \cdots$ and $B(x) = x^2$ (constant is zero). We were only allowed to do this because $[x^0]B(x) = 0$.
				\end{ex}
			\subsubsection{Cartesian Product}
				\begin{defn}
					Let $A$ and $B$ be sets. The \textbf{cartesian product} is defined as:
					\begin{align*}
						A \times B := \{ (a, b) \big| a \in A, b \in B \}
					\end{align*}

					Note that $(a, b)$ is an ordered pair.
				\end{defn}

				\begin{ex}
					Let $A = \{1, 2, 3\}$ and $B = \{x, y\}$. Then $A \times B = \{ (1, x), (2, x), (3, x), (1, y), (2, y), (3, y) \}$.
				\end{ex}

				The cardinality of $A \times B$ is the product of the cardinalities of $A$ and $B$: $|A \times B| = |A||B|$.
		\subsection{The Sum and Product Lemmas}
			Let's consider a bin of two red marbles -- one large marble (denoted $R$), and one small marble (denoted $r$). We define $A$ to be the set of all selections of $\ge 1$ marbles. We know: 
			\begin{align*}
				A &= \{ \{r\}, \{R\}, \{r, R\} \} \\
				w(\sigma) &= |\sigma| \\
				\phi_A(x) &= x + x + x^2 = 2x + x^2
			\end{align*}
			Let's now consider a bin of green marbles -- two large marbles (denoted $G$). We define $B$ to be the set of all selections of $\ge 1$ marbles. We know:
			\begin{align*}
				B &= \{ \{G\}, \{G, G\} \} \\
				w(\sigma) &= |\sigma| \\
				\phi_B(x) &= x + x^2
			\end{align*}
			\subsubsection{Sum Lemma}
				Let $S = A \union B = \{ \{r\}, \{R\}, \{r, R\}, \{G\}, \{G, G\} \}$, and note that $A \intersection B = \emptyset$. What is the generating function for $S$?
				\begin{align*}
					\phi_S(x) = \underbrace{(x + x + x^2)}_{\phi_A(x)} + \underbrace{(x  + x^2)}_{\phi_B(x)}
				\end{align*}
				This works in general.
				
				\begin{theorem}[Sum Lemma]
					We have a set $S$ of objects with weight $w$. Let $A$ and $B$ be partitions of $S$. Then:
					\begin{align*}
						\phi_S(x) = \phi_A(x) + \phi_B(x)
					\end{align*}
				\end{theorem}

				\begin{proof}
					\begin{align*}
						\phi_S(x) = \sum_{\sigma \in S} x^{w(\sigma)} = \underbrace{\sum_{\sigma \in A} x^{w(\sigma)}}_{\phi_A(x)} + \underbrace{\sum_{\sigma \in B} x^{w(\sigma)}}_{\phi_B(x)}
					\end{align*}
				\end{proof}
			\subsubsection{Product Lemma} \lecture{January 18, 2013}
				Let $S = A \times B = \{ (\{r\}, \{G\}), (\{R\}, \{G\}), (\{r, R\}, \{G\}), (\{r\}, \{G, G\}), (\{R\}, \{G, G\}), (\{r, R\}, \{G, G\}) \}$. $S$ is the number of ways of selecting marbles from both bins of marbles. Let $w(\sigma)$ be the number of marbles selected in total.
				\\ \\
				\textbf{Note}: you can think of a generating series as the sum of all objects' $x^{w(\sigma)}$, or you can think of it as the sum of all $a_k x^k$.
				\begin{align*}
					\phi_S(x) &= x^2 + x^2 + x^3 + x^3 + x^3 + x^4 \\
					&= 2x^2 + 3x^3 + x^4 \\
					&= (x + x + x^2)(x + x^2) \\
					&= \phi_A(x) \cdot \phi_B(x)
				\end{align*}
				This works in general.
				\\ \\
				Each term in $\phi_A(x)$ corresponds to an object in $A$, and each term in $\phi_B(x)$ corresponds to an object in $B$. When you find the product, it automatically does the counting for you, due to exponentiation laws.
				\begin{theorem}[Product Lemma]
					Let $A$ be a set where objects have weight $\alpha$, and let $B$ be a set where objects have weight $\beta$. Let $S = A \times B$ be a set of all objects $(a, b) \in S$, where $w[(a, b)] = \alpha(a) + \beta(b)$. Then:
					\begin{align*}
						\phi_S(x) = \phi_A(x) \cdot \phi_B(x)
					\end{align*}
				\end{theorem}

				\begin{proof}
					\begin{align*}
						\phi_S(x) &= \sum{(a, b) \in S} x^{w[(a, b)]} \\
						&= \sum_{(a, b) \in S} x^{\alpha(a) + \beta(b)} \\
						&= \underbrace{\bigg[ \sum_{a \in A} x^{\alpha(a)} \bigg]}_{\phi_A(x)} \cdot \underbrace{\bigg[ \sum_{b \in B} x^{\beta(b)} \bigg]}_{\phi_B(x)}
					\end{align*}
				\end{proof}
			
				\begin{ex}
					Given integers $n \ge 0$ and $k \ge 0$, how many $k$-tuples $(m_1, m_2, \ldots, m_k)$ (where $m_i \ge 0, m_i \in \mathbb{Z}$) exist where $m_1 + m_2 + \ldots + m_k = n$?
					\\ \\
					Suppose $n = 3$ and $k = 2$. You'll have $(0, 3), (1, 2), (2, 1)$, and $(3, 0)$ \textendash{} four possibilities (notice order matters).
					\\ \\
					We have $\mathbb{N}_0 = \{0, 1, 2, 3, \ldots \}, w(a) = a$. Then:
					\begin{align*}
						\phi_{\mathbb{N}_0}(x) = \sum_{n \ge 0} a_n x^n = 1 + x + x^2 + x^3 + \ldots = \frac{1}{1-x}
					\end{align*}

					Now, let $S = \overbrace{\mathbb{N}_0 \times \mathbb{N}_0 \times \ldots \times \mathbb{N}_0}^{\mathbb{N}_0 \text{ appears } k \text{ times}}$. The objects in $S$ are sequences of form $(m_1, m_2, \ldots, m_k)$, where $m_i \ge 0, m_i \in \mathbb{Z}$. We define $w[(m_1, m_2, \ldots, m_k)] = m_1 + m_2 + \ldots + m_k$.
					\begin{align*}
						\phi_S(x) = [\phi_{\mathbb{N}_0}(x)]^k = \frac{1}{(1-x)^k} \text{ by the product lemma}
					\end{align*}

					Therefore, the number of $k$-tuples $(m_1, m_2, \ldots, m_k$ (where $m_i \ge 0, m_i \in \mathbb{Z}$) where $m_1 + m_2 + \ldots + m_k = n$ is given by $[x^n]\frac{1}{(1 - x)^k}$. But we cannot figure out the coefficient. Let's leave generating functions for now, and find a combinatorial proof for this.
				\end{ex}

				When you use the product lemma, you need to ensure:
				\begin{enumerate}
					\item Ensure it's used on something found by taking the cartesian product.
					\item The weight is equal to the sum of the weights.
				\end{enumerate}
				
				\begin{ex}
					Let $n = 5$ and $k = 3$. We have:

					\begin{center}
						(1, 2, 2): \textbullet{} | \textbullet{} \textbullet{} | \textbullet{} \textbullet{} \\
						(2, 3, 0): \textbullet{} \textbullet{} | \textbullet{} \textbullet{} \textbullet{} |
					\end{center}

					We have $n$ dots and $k-1$ intervals, which gives us:
					\begin{align*}
						{n + (k - 1) \choose k - 1} = {7 \choose 2} \text{ (in this case) }
					\end{align*}

					We always want two bars because $k = 3$. In general, we always want $k - 1$ bars. We want $n$ dots. The combination above is formed by the need to place $k - 1$ bars among all of the positions.
				\end{ex}
				By combining the previous two examples, we have proven the following theorem.
				\begin{theorem}[Theorem 1.6.5]
					\begin{align*}
						[x^n] \frac{1}{(1 - x)^k} = {n + k - 1 \choose k - 1}
					\end{align*}
				\end{theorem}

				\begin{defn}
					Some integer $n$ has \textbf{composition} $(m_1, m_2, \ldots, m_k)$ if $m_1 + m_2 + \ldots + m_k = n$, where $m_i \ge 1, m_i \in \mathbb{Z}$. Note that the only difference from earlier is that $m_i \ge 1$ instead of $m_i \ge 0$.
				\end{defn}
				
				\begin{ex}
					We have $\mathbb{N} = \{1, 2, 3, \ldots \}, w(a) = a$, which gives us:
					\begin{align*}
						\phi_{\mathbb{N}}(x) = x + x^2 + x^3 + \ldots = x(1 + x + x^2 + \ldots) = \frac{x}{1 - x}
					\end{align*}

					Note that $\phi_{\mathbb{N}}(x)$ is missing the object of size zero (the constant term).
					\\ \\
					Let $S = \overbrace{\mathbb{N} \times \mathbb{N} \times \ldots \times \mathbb{N}}^{k \text{ occurrences of } \mathbb{N} \text{ here}}$. We also define the weight as $w[(m_1, m_2, \ldots, m_k)] = m_1 + m_2 + \ldots + m_k$. Then:
					\begin{align*}
						\phi_S(x) = [\phi_{\mathbb{N}}(x)]^k = \frac{x^k}{(1 - x)^k}
					\end{align*}
					\lecture{January 21, 2013}
					We're interested in finding $[x^k]x^k \frac{1}{(1 - x)^k}$.
					\begin{align*}
						[x^n]x^k \frac{1}{(1 - x)^k} &= [x^{n-k}] \frac{1}{(1 - x)^k} \\
						&= {(n - k) + k - 1 \choose k - 1} \\
						&= {n - 1 \choose k - 1}
					\end{align*}
				\end{ex}

				\begin{ex}
					We now want to know the number of compositions of $n$ (where $n \ge 1$) where we have an arbitrary number of parts. We have: $S = \mathbb{N} \union \mathbb{N}^2 \union \mathbb{N}^3 \union \mathbb{N}^4 \union \mathbb{N}^5 \union \ldots$. The weight function is the same as before \textendash{} the sum of the parts. That gives us:
					\begin{align*}
						\phi_S(x) = \sum_{k \ge 1} \phi_{\mathbb{N}^k}(x) = \sum_{k \ge 1} \bigg( \frac{x}{1 - x} \bigg)^k
					\end{align*}

					We want:
					\begin{align*}
						[x^n]\sum_{k \ge 1} \bigg( \frac{x}{1 - x} \bigg)^k
					\end{align*}
					Which is a composition of:
					\begin{align*}
						A(y) &= y + y^2 + y^3 + \ldots = y(1 + y + y^2 + \ldots) = \frac{y}{1 - y} \\
						B(y) &= \frac{x}{1 - x}
					\end{align*}
					The composition is $A(B(x))$, but we need to be careful before using compositions. We can easily check that the constant of $B(x)$ is not zero, as required:
					\begin{align*}
						B(x) = \frac{x}{1 - x} = x(1 + x + x^2 + \ldots) \implies [x^0]B(x) = 0
					\end{align*}
					The composition is well-defined. So, we have:
					\begin{align*}
						\sum_{k \ge 1} \bigg( \frac{x}{1 - x} \bigg)^k = \frac{\frac{x}{1 - x}}{1 - \frac{x}{1 - x}} = \frac{x}{1 - 2x}
					\end{align*}
					We still need to figure out the coefficient $[x^n]\frac{x}{1 - 2x}$.
					\begin{align}
						[x^n] \frac{x}{1 - 2x} &= [x^{n - 1}] \frac{1}{1 - 2x} \\
						&= [x^{n - 1}](1 + y + y^2 + \ldots) \\
						&= [x^{n - 1}](1 + (2x) + (2x)^2 + \ldots) \\
						&= 2^{n - 1}
					\end{align}
					(1) is a common, useful trick. We eliminate any instances of $x$ that are simply multiplied by something, by reducing the power of the coefficient we're looking for. (4) is true since every term is in the form $(2x)^k$.
				\end{ex}

				We might be able to avoid generating functions in some cases by using a combinatorial proof. Now, we'll look at an arbitrary, convoluted example where a combinatorial proof would not suffice. This aims to show that generating functions are solid and can apply in many more situations than combinatorial proofs.
				
				\begin{ex}
					We want to find the number of compositions of $n$ with $2k$ parts, where the first $k$ parts are $\le 5$ and the last $k$ parts are $\ge 3$. 
					\\ \\
					A specific instance of this problem is when $n = 22$ and $k = 3$, we have:
					\begin{align*}
						(\underbrace{\dots}_{\le 5}&, \underbrace{\dots}_{\ge 3}) \\ \\
						(\underbrace{1, 3, 5}_{\le 5}&, \underbrace{3, 6, 4}_{\ge 3})
					\end{align*}
					It's useful to think about what you're counting. In this case, you're counting the $k$-tuples, so this is the cartesian product of six sets (the first three being the set of positive integers $\le 5$ and the last three being the set of integers $\ge 3$).
					\begin{align*}
						\mathbb{N}_{\le 5} &= \{1, 2, 3, 4, 5\} \\
						\mathbb{N}_{\ge 3} &= \{3, 4, 5, \ldots\}
						\\ \\
						S &= (\mathbb{N}_{\le 5})^k \times (\mathbb{N}_{\ge 3})^k
					\end{align*}
					The weight is the sum of the six individual numbers, since $\mathbb{N}_{\le 5}$ and $\mathbb{N}_{\ge 3}$'s weights are both defined by $w(a) = a$, which gives us:
					\begin{align*}
						\phi_{\mathbb{N}_{\le 5}}(x) &= x + x^2 + x^3 + x^4 + x^5 \\
						\phi_{\mathbb{N}_{\ge 3}}(x) &= x^3 + x^4 + x^5 + \ldots
					\end{align*}

					\lecture{January 23, 2013}
					We can expand the generating functions $\phi_{\mathbb{N}_{\le 5}}(x)$ and $\phi_{\mathbb{N}_{\ge 3}}(x)$.
					\begin{align*}
						\phi_{\mathbb{N}_{\le 5}}(x) &= x + x^2 + x^3 + x^4 + x^5 \\
						&= x(1 + x + x^2 + x^3 + x^4) \\
						&= x \bigg( \frac{1 - x^5}{1 - x} \bigg)
						\\ \\
						\phi_{\mathbb{N}_{\ge 3}}(x) &= x^3 + x^4 + x^5 + \ldots \\
						&= x^3(1 + x + x^2 + \ldots) \\
						&= \frac{x^3}{1 - x}
					\end{align*}

					Now, we have $S = (\mathbb{N}_{\le 5})^k \times (\mathbb{N}_{\ge 3})^k$ with weight function $w[(a_1, \ldots, a_2k)] = a_1 + a_2 + \ldots + a_{2k}$. By the product lemma, we get:
					\begin{align*}
						\phi_S(x) &= x^k \bigg(\frac{1 - x^5}{1 - x}\bigg)^k \bigg( \frac{x^3}{1 - x} \bigg)^k \\
						&= x^{4k}(1 - x^5)^k (1-x)^{-2k}
					\end{align*}
					We're still interested in finding $[x^n]\phi_S(x)$. This becomes tedious, but the general idea is:
					\begin{align*}
						[x^n] \underbrace{x^{4k}}_{\text{collapse into } [x^n] \text{ to become } [x^{n - 4k}]} \cdot \underbrace{(1 - x^5)^k}_{\text{expand with binomial theorem}} \cdot \underbrace{(1 - x)^{-2k}}_{\text{ use formula } [x^n]\frac{1}{(1 - x)^k}}
					\end{align*}
					After much tedious work, you'll discover that the final result is:
					\begin{align*}
						[x^n] \phi_S(x) = \sum_{i = 0}^{\lfloor \frac{n - 4k}{5} \rfloor} (-1)^i {k \choose i} {n - 5i - 2k - 1 \choose 2k - 1}
					\end{align*}

					This couldn't have been found with a combinatorial proof. That's why generating functions are powerful \textendash{} they work even in convoluted situations like this one.
				\end{ex}

				\begin{ex}
					We're interested in the number of compositions of $n$ where all parts are odd. There can be an arbitrary number of parts.
					\\ \\
					For example, with $n = 5$, the five solutions are:
					\begin{align*}
						\{ (1, 1, 1, 1, 1), (5), (3, 1, 1), (1, 3, 1), (1, 1, 3) \}
					\end{align*}
					We'll define $n = 0$ to have a unique composition (), the empty tuple.
					\\ \\
					We have $\mathbb{N}_{\text{odd}} = \{1, 3, 5, 7, \ldots\}$, with its weight function defined to be $w(a) = a$ for any $a \in \mathbb{N}_{\text{odd}}$. That gives us its generating function:
					\setcounter{equation}{0}
					\begin{align}
						\phi_{\mathbb{N}_{\text{odd}}}(x) &= x + x^3 + x^5 + x^7 + \ldots \\
						&= x(1 + x^2 + x^4 + x^6 + \ldots) \\
						&= x(1 + y + y^2 + y^3 + \ldots) \text{ where } y = x^2 \\
						&= \frac{x}{1 - y} \\
						&= \frac{x}{1 - x^2}
					\end{align}

					(3) is an acceptable composition because $x^2$ has a constant term of zero.
					\\ \\
					Now, we have that $S = () \cup_{k \ge 1} \mathbb{N}_{\text{odd}}^k$. By the sum and product lemmas, the generating function for $S$ is:
					\setcounter{equation}{0}
					\begin{align}
						\phi_S(x) &= 1 + \sum_{k \ge 1} \phi_{\mathbb{N}_{\text{odd}}}^k(x) \\
						&= 1 + \sum_{k \ge 1} \bigg( \frac{x}{1 - x^2} \bigg)^k \\
						&= \sum_{k \ge 0} \bigg( \frac{x}{1 - x} \bigg)^k \\
						&= 1 + z + z^2 + z^3 + \ldots \text{ where } z = \frac{x}{1 - x^2} \\
						&= \frac{1}{1 - z} \\
						&= \frac{1}{1 - \frac{x}{1 - x^2}} \\
						&= \frac{1 - x^2}{1 - x - x^2}
					\end{align}

					(4) is a valid composition because $\frac{x}{1 - x^2}$ has a constant term of zero.
					\\ \\
					What is $[x^n] \frac{1 - x^2}{1 - x - x^2}$? (Don't think of assigning values to $x$. Just focus on cancelling them out.)
					\\ \\
					\begin{align*}
						\frac{1 - x^2}{1 - x - x^2} &= \sum_{n \ge 0} a_n x^n \\
						1 - x^2 &= (1 - x - x^2)(a_0 + a_1x + a_2x^2 + \ldots) \\
						&= a_0 + a_1x + a_2x^2 + a_3x^3 + \ldots - a_0x -a_1x^2 - a_2x^3 - a_3x^4 - \ldots -a_0x^2 - a_1x^3 - a_2x^4 - a_3x^5 - \ldots \\
						&= a_0 + (a_1 - a_0)x + (a_2 - a_1 - a_0)x^2 + (a_3 - a_2 - a_1)x^3 + \ldots
					\end{align*}

					We know $a_0 = 1$ and $(a_1 - a_0) = 0$, and so on. We can express a recurrence relation for this.
					
					\begin{align*}
						a_0 &= 1 \\
						a_1 - a_0 = 0 \implies a_1 &= 1 \\
						a_2 - a_1 - a_0 = -1 \implies a_2 &= -1 + a_1 + a_0 = 1 \\
						a_3 - a_2 - a_1 = 0 \implies a_3 &= a_1 + a_2 = 2 \\
						a_4 &= a_3 + a_2 = 3 \\
						a_5 &= a_4 + a_3 = 5 \\
						\vdots&
					\end{align*}

					For all $n \ge 3$, we have that $a_n = a_{n - 1} + a_{n - 2}$. This defines the \textbf{fibonacci numbers}.
				\end{ex}
				
				\begin{defn}
					The \textbf{golden ratio} is a pair of integers $a$ and $b$ such that $\frac{a + b}{a} = {a}{b}$. It's a ratio that comes up in nature a lot, and is aesthetically pleasing.
				\end{defn}

				You can approximate the golden ration with the Fibonacci numbers. $\frac{a_n}{a_{n - 1}}$ approaches the golden ratio as $n \to \infty$.
				
				\begin{ex}
					Let $S_n$ be the set of compositions into odd numbers of $n$. $|S_n| = |S_{n - 1}| + |S_{n - 2}|$ for $n \ge 2$. $S_n$ is partitioned by $S_{n - 1}$ and $S_{n - 2}$. Is there a bijection between $S_n$ and $S_{n - 1} \cup S_{n - 2}$?
					\\ \\
					Let's look at the case where $n = 5$. We have: \lecture{January 25, 2013}
					\begin{align*}
						\overbrace{\{ (1, 1, 1, 1, 1), (3, 1, 1), (1, 3, 1), (1, 1, 3), (5) \}}^{S_n} \\
						\{ \underbrace{(1, 1, 1, 1), (3, 1), (1, 3)}_{S_{n - 1}}, \underbrace{(1, 1, 1), (3)}_{S_{n - 2}} \} 
					\end{align*}
					Notice that each element in $S_{n - 1}$ and $S_{n - 2}$ can be obtained from $S_n$ by either removing the last element entirely, or by subtracting 2 from the last element. That's the bijection we're looking for, which we will now define more formally as $f : S_n \to S_{n - 1} \cup S_{n - 2}$:
					\begin{align*}
						f(a_1, \ldots, a_k) = \begin{cases}
							(a_1, \ldots, a_{k - 1}) & a_k = 1 \\
							(a_1, \ldots, a_k - 2) & a_k \ge 3
						\end{cases}
					\end{align*}
					Note that $f(a_1, \ldots, a_k)$ is undefined for $1 < k < 3$, but that's okay because we're only interested in odd natural numbers. 
					\\ \\
					We have to somehow show that $f$ actually is a bijection. We can do that by constructing its inverse. In our case, the inverse is simply providing an inverse procedure.
					\\ \\
					Therefore, $|S_n| = |S_{n - 1}| + |S_{n - 2}|$, which is a general property of bijections.
				\end{ex}
	
	\section{Binary Strings}
		\begin{defn}
		A \textbf{binary string} is a string composed of 0s and 1s. For example, $011001110011$ is a binary string.
		\end{defn}

		\begin{defn}
			We define the operation of \textbf{concatenation of two binary strings} $a_1, a_2$ as $a_1 a_2$. For example, if $a_1 = 110$ and $a_2 = 011$, the concatenation $a_1 a_2 = 110011$.
		\end{defn}

		\begin{defn}
			The \textbf{concatenation $AB$ of two sets} ($A$ and $B$) of binary strings is defined as all possible strings formed as concatenations of one string from set $A$ followed by one string of set $B$. For example, if $A = \{ 011, 11 \}$ and $B = \{10, 0\}$, then $AB = \{ 01110, 0110, 1110, 110 \}$.
		\end{defn}

		\begin{defn}
			Suppose $A$ is a set of strings. Then $A^k = \underbrace{AAAAA \ldots A}_{k \text{ times}}$.
		\end{defn}

		\begin{defn}
			We define $A^\star = \{ \epsilon \} \cup A \cup A^2 \cup A^3 \cup \ldots$, where $\epsilon$ represents the empty string.
		\end{defn}

		\begin{ex}
			What is $\{ 0, 1 \}^\star$? It's the set of \emph{all} binary strings.
		\end{ex}

		\begin{ex}
			What's $\{ 0, 1 \}^5$? $\{ 0, 1 \}^5 = \{ 0, 1 \} \{ 0, 1 \} \{ 0, 1 \} \{ 0, 1 \} \{ 0, 1 \}$. $\{0, 1\}^5$ is the set of all binary strings of length 5.
		\end{ex}

		\begin{ex}
			Describe the set $S$ of all binary strings with no three consecutive 1s. For example: $a = 011000\underline{111}01001 \not \in S$, $b = 0\underline{111}10 \not \in S$, but $c = 0110010011011 \in S$. We aim to express $S$ using expression $\star$ concatenation.
			\\ \\
			Let's take a second look at $c$.
			\begin{align*}
				c = \underbrace{0,110,0,10,0,110}_{(1)},\underbrace{11}_{(2)} \in S
			\end{align*}
			
			\begin{enumerate}
				\item A sequence of at most two ones followed by zero, repeated: $\{ 0, 10, 110 \}^\star$
				\item A sequence of zero, one, or two ones: $\{ \epsilon, 1, 11 \}$.
			\end{enumerate}

			Therefore, $S = \{0, 10, 110\}^\star \{ \epsilon, 1, 11 \}$. Note that $\epsilon$ is an element of $\{ \text{anything} \}^\star$.
		\end{ex}

		\begin{defn}
			A \textbf{block} in a binary string is equal to the maximum sequence of 0s or 1s.
		\end{defn}
		
		\begin{ex}
			List all the blocks in the string $1100011110110101$.
			\begin{align*}
				\underbrace{11} \underbrace{000} \underbrace{1111} \underbrace{0} \underbrace{11} \underbrace{0} \underbrace{1} \underbrace{0} \underbrace{1}
			\end{align*}
		\end{ex}

		\begin{ex}
			Describe the set $S$ of strings with no blocks of length 2. For example, $a = 10\underline{11}000101 \not \in S$, but $b = 1110111000101000 \in S$.

			Let's take another look at $b$.
			\begin{align*}
				b = \underbrace{111}_{(1)},\underbrace{0111,0001,01}_{(2)},\underbrace{000}_{(3)} \in S
			\end{align*}

			\begin{enumerate}
				\item A sequence of ones, of any length other than length 2: $\{ \epsilon, 1, 111, 1111, 11111, \ldots \}$.
				\item A sequence of zeroes (of any non-zero length other than length 2), followed by a sequence of ones (of any non-zero length other than length 2), repeated: $[ \{0, 000, 0000, \ldots \} \{1, 111, 1111, \ldots \} ]^\star$.
				\item A sequence of zeroes, of any length other than length 2: $\{ \epsilon, 0, 000, 0000, 00000, \ldots \}$.
			\end{enumerate}

			It's important to note that $\{ 1, 111, 1111, \ldots, 0, 000, 0000, \ldots \}^\star$ does not work because you could pick two zeroes or two ones consecutively, which is not allowed. You have to be very careful that you don't generate strings that you don't want to generate.
		\end{ex}

		\subsection{Generating Functions for Strings}
			We'd like to have something like the product lemma but for strings.
			\\ \\
			Let $S$ be the set of binary strings. Then the generating function for $S$ is defined as:
			\begin{align*}
				\phi_S(x) = \sum_{n \ge 0} a_n x^n \text{ where } a_n \text{ is the number of strings of length } n \in S
			\end{align*}

			\begin{ex} \lecture{January 28, 2013}
				Let $A = \{0, 01\}$ and $B = \{0, 10\}$. Their generating functions are $\phi_A(x) = \phi_B(x) = x + x^2$. What is the generating function for $AB$?
				\\ \\
				It's easy to see that $AB = \{00, 010, 0110\}$, which gives us $\phi_{AB}(x) = x^2 + x^3 + x^4$. Also note that:
				\begin{align*}
					\phi_A(x) \cdot \phi_B(x) &= (x + x^2) \cdot (x + x^2) \\
					&= x^2 + 2x^3 + x^4 \\
				\end{align*}
				Therefore, in this case, $\phi_{AB}(x) \not = \phi_A(x) \cdot \phi_B(x)$. Why is this? $A \times B = \{ (0, 0), (01, 0), (0, 10), (0110) \}$. Note that the string for $(01, 0)$ and $(0, 10)$ is the same ($010$), but it's only going to be included in $AB$ once. That's where the trouble is.
			\end{ex}

			\begin{defn}
				$A, B$ is \textbf{ambiguous} if all three of these properties hold:
				\begin{itemize}
					\item $(a_1, b_1), (a_2, b_2) \in A \times B$
					\item $(a_1, b_1) \not = (a_2, b_2)$
					\item $a_1b_1 = a_2b_2$
				\end{itemize}
			\end{defn}
			In the previous example, $(0, 10) (01, 0) \in A \times B$ and $(0, 10) \not = (10, 0)$, but $010 = 010$.
			\\ \\
			Note that the definition of ambiguity is equivalent to saying that $|A \times B| \not = |AB|$.

			\begin{theorem}[Product Lemma for Strings]
				Let $A, B$ be sets of binary strings. If $A, B$ are unambiguous, then $\phi_{AB}(x) = \phi_A(x) \cdot \phi_B(x)$.
			\end{theorem}

			\begin{proof}
				In order to prove this theorem, we need to show that under the given assumptions, $|AB| = |A \times B|$, which we can show by showing there is a bijection.
				\\ \\
				Let $S_n$ be the set of strings of length $n$ of $AB$, and let $T_n$ be the set of pairs $(a, b) \in A \times B$ such that $length(a) + length(b) = n$. Our claim is that $f[(a, b)] = ab, f : T_n \to S_n$ is a bijection. 
				\\ \\
				In order to prove that $T_n \to S_n$ is a bijection, two things must be proven. We have to show that it is both onto and injective.
				\begin{enumerate}
					\item Onto. Every set in $S_n$ is in the form $ab$ where $a \in A$ and $b \in B$ such that $ab = f[(a, b)]$.
					\item Injective. Let $(a_1, b_1), (a_2, b_2) \in T_n$. Suppose:
						\begin{align*}
							\underbrace{f[(a_1, b_1)]}_{a_1b_1} = \underbrace{f[(a_2, b_2)]}_{a_2b_2}
						\end{align*}
						Then since $A,B$ is unambiguous, $(a_1, b_1) = (a_2, b_2)$.
				\end{enumerate}

				Using the claim we just proved, $|S_n| = |T_n|$. By the product lemma (on generating functions in general), we have:
				\begin{align*}
					|S_n| = |T_n| = [x^n] \phi_{A \times B}(x) = [x^n] \phi_A(x) \cdot \phi_B(x)
				\end{align*}
			\end{proof}

			\begin{theorem}[Sum Lemma for Strings]
				Let $S$ be a set of binary strings. Let $A, B$ be partitions of $S$. Then $\phi_S(x) = \phi_A(x) + \phi_B(x)$.
			\end{theorem}

			The proof of the sum lemma for strings is the same as the proof of the sum lemma for sets in general.

			\begin{defn}
				An expression is \textbf{unambiguous} if there exists a unique way of writing the string according to the expression.
			\end{defn}

			For example: $\{0, 10, 110\}^\star \{\epsilon, 1, 11\}$ is the set of strings with no more than two consecutive ones. Let's examine the arbitrary string $0001101011001$. If we split the string after every zero, we get $0,0,0,110,10,110,0,1$, which is a unique way of breaking down the string.

			\begin{proposition}
				Suppose $A^\star$ is unambiguous and $\epsilon \not \in A$. Then $\phi_A(x) = \frac{1}{1 - \phi_A(x)}$.
				\label{proposition:unambiguousstar}
			\end{proposition}

			\begin{proof}
				We have $A^\star = \{\epsilon\} \cup A \cup A^2 \cup A^3 \cup A^4 \cup \ldots$. What is $\phi_{A^k}(x)$?
				\begin{align*}
					\phi_{A^k}(x) &= [\phi_A(x)]^k \text{ by the product lemma for strings} \\
					&= 1 + \phi_A(x) + \phi_{A^2}(x) + \phi_{A^3}(x) + \ldots \text{ by the sum lemma} \\
					&= 1 + \phi_A(x) + [\phi_A(x)]^2 + [\phi_A(x)]^3 + \ldots \\
					&= 1 + y + y^2 + y^3 \text{ by letting } y = \phi_A(x) \\
					&= \frac{1}{1 - y} \\
					&= \frac{1}{1 - \phi_A(x)}
				\end{align*}

				Note that the composition used in this proof is valid because $\epsilon \not \in A$, so the constant of $\phi_A(x)$ is zero as required. That is, $[x^0]\phi_A(x) = 0$ makes the composition acceptable.
			\end{proof}
		\subsection{Counting Strings} \lecture{January 30, 2013}
			\begin{ex}
				Let $S$ be the set of binary strings with no three consecutive ones (111). We can express $S$ as $S = \{ 0, 10, 110 \}^\star \{\epsilon, 1, 11\}$. You can verify that this is unambiguous.
				\\ \\
				We have:
				\begin{align*}
					\phi_{\{0, 10, 110\}}(x) &= x + x^2 + x^3 \\
					\phi_{\{0, 10, 110\}^\star}(x) &= \frac{1}{1 - x - x^2 - x^3} \text{ by \ref{proposition:unambiguousstar}} \\
					\phi_{\{\epsilon, 1, 11\}}(x) &= 1 + x + x^2 \\ \\
					\phi_S(x) &= \phi_{\{0, 10, 110\}^\star}(x) \cdot \phi_{\{\epsilon, 1, 11\}}(x) \text{ by the product lemma for strings} \\
					&= \frac{1 + x + x^2}{1 - x - x^2 - x^3}
				\end{align*}
			\end{ex}

			\begin{ex}
				Let $S$ be the set of binary strings where no block has length 2. Recall from before:
				\begin{align*}
					S &= ABC \\
					A &= \{\epsilon, 1, 111, 1111, \ldots\} \\
					B &= (\{0, 000, 0000, \ldots\}\{1, 111, 1111, \ldots\})^\star
				\end{align*}

				For the sake of convenience, we'll define $f$ as follows.
				\begin{align*}
					f &= x + x^3 + x^4 + x^5 + \ldots \\
					&= x + x^2 + x^4 + x^5 + \ldots - x^2 \text{ by adding and removing } x^2 \\
					&= x(1 + x + x^2 + \ldots) - x^2 \\
					&= \frac{x}{1 - x} - x^2
				\end{align*}

				Now, we can easily state the generating functions for $A$, $B$, and $C$:
				\begin{align*}
					\phi_A(x) &= 1 + f \\
					\phi_B(x) &= \frac{1}{1 - f^2} \\
					\phi_C(x) &= 1 + f
				\end{align*}

				We can now use the product lemma on $ABC$ to find $\phi_S(x)$.
				\begin{align*}
					\phi_S(x) &= \phi_A(x) \cdot \phi_B(x) \cdot \phi_C(x) \\
					&= \frac{(1 + f)^2}{1 - f^2} \\
					&= \frac{(1 + f)(1 + f)}{(1 + f)(1 - f)} \\
					&= \frac{1 + f}{1 - f} \\
					& \text{\vdots} \\ 
					&= \frac{1 - x^2 + x^3}{1 - 2x + x^2 - x^3}
				\end{align*}
			\end{ex}
			\subsubsection{Recursive String Definitions}
				There's another way of constructing binary strings: recursive definitions.
				\begin{ex}
					Let $S$ be the set of all binary strings. We can define $S$ recursively as $S = \{0, 1\} \cup \{ \epsilon \}$. Whenever you write an expression like this, you should take great care to ensure that both sides are actually equal.
					\\ \\
					One might attempt to describe $S$ as $\{ \epsilon, 0, 1 \}S$, however that is ambiguous. $S = \{ 0, 1 \} \cup \{ \epsilon \}$ is unambiguous. Since there are two terms of length 1, and applying the sum \& product lemmas, we get:
					\begin{align*}
						\phi_S(x) &= 2x \cdot \phi_S(x) + 1 = \frac{1}{1 - 2x} \\
						[x^n]\phi_S(x) &= [x^n] \frac{1}{1 - 2x} \\
						&= [x^n](1 + 2x + (2x)^2 + (2x)^3 + \ldots
					\end{align*}
				\end{ex}
	\newpage
	\section{Clicker Questions}
		\begin{itemize}
			\item \lecture{January 21, 2013} How many bijections are there from $\{1, \ldots, n\}$ to $\{1, \ldots, n\}$? $n!$, since you can't map two elements in the first set to the same one element in the second set.
			\item Let $S$ be a set of objects with weight function $w$. Suppose that $S = A \union B$. Does $\phi_S(x) = \phi_A(x) + \phi_B(x)$? No, not always. The sum lemma only applies when $A$ and $B$ are partitions of $S$.
			\item Let $S = \{(a, b) | a, b \ge 1 \in \mathbb{Z}\}, w[(a, b)] = a - b$. What is the generating function for $S$? There is no generating function for $S$, since it is not a power series. There are an infinite number of pairings \textendash{} that is, there are an infinite number of objects with the same weight.
			\item Does $x^2 - x^4 + x^6 - x^8 + x^10 - \ldots$ have an inverse? No. There is no constant term that isn't equal to zero, therefore it does not have an inverse.
			\item Is a composition of $A(x) = 1 + x + x^2 + x^3 + \ldots$ and $B(x) = \frac{1}{1 - x}$ a power series? No. $B(x)$ must have a constant term that is not zero in order for a composition to be acceptable.
			\item \lecture{January 30, 2013} What is the power series $x^2 + x^3 + x^4 + x^5 + \ldots$ equal to? It is equal to $\frac{x^2}{1 - x}$, since:
				\begin{align*}
					x^2 + x^3 + x^4 + x^5 + \ldots &= x^2(1 + x + x^2 + \ldots) = x^2 \bigg(\frac{1}{1 - x}\bigg) = \frac{x^2}{1 - x}
				\end{align*}
			\item Is $A(B(x))$ a power series for $A(x) = 1 + x + x^2 + \ldots$ and $B(x) = \frac{1}{1 - x}$? No! We can only get a power series if $[x^0]B(x) = 0$, but $B(x) = \frac{1}{1 - x} = 1 + x + x^2 + \ldots$.
			\item What is the coefficient of $x^3$ for $\frac{1}{(1 - x)^7}$? The coefficient is ${9 \choose 2}$. This could be found using the formula:
				\begin{align*}
					[x^n] \frac{1}{(1 - x)^k} = { n + k - 1 \choose k - 1 }
				\end{align*}
			\item The number of ways of spending exactly \$237 to buy oranges (\$1 each) and/or mangos (\$2 each) is\dots? It's $[x^{237}](1 + x + x^2 + x^3 + \ldots)(1 + x^2 + x^4 + x^6 + \ldots)$.
				\\ \\
				You can represent this as a pair, $(a, b)$, where $a$ is the number of oranges and $b$ is the number of mangos. Define a weight function $w[(a, b)] = a + 2b$. We also have $S = \{ (a, b) \}, A = \{ 0, 1, \ldots \}, B = \{ 0, 1, \ldots \}$. Also, define weight functions for $A$ and $B$: $w(a) = a, w(b) = 2b$. The generating series are $\phi_A(x) = 1 + x + x^2 + \ldots$ and $\phi_B(x) = 1 + x^2 + x^4 + \ldots$. Finally, use the product lemma on $\phi_A(x) \cdot \phi_B(x)$.

			\item The set of binary strings without consecutive zeroes can be described as\dots? The set can be described as $(\{\epsilon, 0\}\{ 1 \})^\star \{ \epsilon, 0 \}$.
				\\ \\
				We have to make sure this expression does not generate anything that is not in the set. Then we think about how to generate any arbitrary string contained in the set we're trying to express. There is a unique way to generate each string with the expression.
		\end{itemize}
\end{document}
