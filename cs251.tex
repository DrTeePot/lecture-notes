\documentclass[]{article}
\usepackage[margin = 1.5in]{geometry}
\setlength{\parindent}{0in}
\usepackage{amsfonts}
\usepackage{amssymb}
\usepackage{hyperref}
\usepackage[T1]{fontenc}
\usepackage{ae,aecompl}

\setlength{\marginparwidth}{1.5in}
\newcommand{\lecture}[1]{\marginpar{{\footnotesize $\leftarrow$ \underline{#1}}}}

\begin{document}

	\title{\bf{CS 251: Computer Organization and Design}}
	\date{Winter 2013, University of Waterloo \\ \center Notes written from Igor Ivkovi\'c's lectures.}
	\author{Chris Thomson}
	\maketitle
	\newpage

	\section{Introduction \& Course Structure} \lecture{January 7, 2013}
		The grading scheme is 50\% final, 30\% midterm, and 20\% (+ 1\%) assignments. There are five assignments, plus a bonus `assignment 0' which is worth an extra 1\%. Assignments will typically be due on Fridays just before noon. 
		\\ \\
		The course aims to discuss the architecture and organization of physical computer systems. It will explain theoretical and practical concepts relevant to the structure and design of the system, which will enable students to predict the behavior and performance of programs on real machines.
		\\ \\
		See the \href{https://www.student.cs.uwaterloo.ca/~cs251/W13/outline.html}{course outline} for more information.

\end{document}